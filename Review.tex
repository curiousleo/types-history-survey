\RequirePackage[l2tabu, orthodox]{nag}
\documentclass[12pt,toc=bibliography,numbers=noendperiod,
               footnotes=multiple,twoside]{scrartcl}
\usepackage{fixltx2e} % LaTeX patches, \textsubscript
\usepackage{microtype}
\usepackage{cmap} % fix search and cut-and-paste in Acrobat
\usepackage{ifthen}
\usepackage[oldstylenums,largesmallcaps,easyscsl]{kpfonts}
\usepackage[LGR,T1]{fontenc}
\usepackage[utf8]{inputenc}
\usepackage{textalpha}
\usepackage[british]{babel}
\usepackage{csquotes}
\usepackage{booktabs}
\usepackage{bussproofs}
\usepackage{paralist}
\usepackage[autocite=footnote,citestyle=authoryear-comp,bibstyle=authoryear,
            dashed=false,isbn=false,doi=false,backend=biber]{biblatex}
\usepackage[bookmarks,hidelinks]{hyperref}
\usepackage[noabbrev,capitalise]{cleveref}

%%% Custom LaTeX preamble
% serif, non-bold headings:
\addtokomafont{disposition}{\rmfamily}
\addtokomafont{descriptionlabel}{\rmfamily}
\addtokomafont{pageheadfoot}{\itshape}
% section numbering up to subsection
\setcounter{secnumdepth}{2}

% hyperlinks
\urlstyle{same} % normal text font (alternatives: tt, rm, sf)
\hypersetup{
  pdftitle={The Curry-Howard correspondence},
  pdfauthor={Leonhard Markert (lm510), Emmanuel College}
}
\title{The Brouwer-Heyting-Kolmogorov-Curry-Howard-Fey correspondence,\\\emph{or}: the formulae-as-types, propositions-as-sets interpretation}
\author{Leonhard Markert}

\addbibresource{Bibliography.bib}
% \pagestyle{headings}

\hyphenation{com-po-nent tran-si-tion-ing}

%%% Body
\begin{document}

\maketitle

\section{Introduction}

% In order to make automated reasoning possible, we need to represent our propositions and proofs in a way that computers can read and manipulate. Lambda calculi have been used to build such representations with great success. I will review a selection of influential publications on the correspondence between various flavours of the lambda calculus and the propositions and proofs they allow us to represent. I will focus on typed lambda calculi à la Church in this survey.

XXX Need to introduce notation here: what is a type, a sort, a type judgement, a context? Also natural deduction primer: entailment, tree structure, axioms and rules.

\section{Early work on type systems and the Curry-Howard correspondence}

XXX Blurb on Schönfinkel and Church in the 20s and 30s?

\subsection{Howard on the Curry-Howard correspondence}

William Howard coined the slogan \enquote{formulae as types} in a set of notes privately circulated in 1969. They appeared as an essay in a book dedicated to Haskell B. Curry on the occasion of his 80th birthday in 1980 \autocite{howard_formulae-as-types_1980}. By this point, the idea of a correspondence between types and propositions had become widely accepted and research into this suprising connection was well under way. Still, this essay is often cited as the publication that originated what became known as the Curry-Howard correspondence (or isomorphism\footnote{XXX explain why it's not an isomorphism}).

Howard starts by noting that the rule which introduces an implication on the right-hand side in the sequent calculus is similar to \(\lambda\)-abstraction, and \emph{modus ponens} (the elimination rule for implication on the right-hand side) resembles application in the \(\lambda\)-calculus.

This simple
% (positive implicational propositional)
logic is then extended to include negation as functions with the empty type as their codomain, conjunction as pairing, and disjunction as what would today be called sum types. These corresponding notions from logic and type theory are listed in \cref{tab:howard}. The second part of the essay is concerned with constructing a set-theoretic interpretation of Heyting arithmetic including existential and universal quantifiers.

\subsection{De Bruijn's Automath project}

In the same book that Howard published his notes in, Nicolaas de Bruijn describes his project \enquote{to develop a system of writing entire mathematical theories in such a precise fashion that verification of the correctness can be carried out by formal operations on the text}, with the express intention to have computers verify proofs \autocite{de_bruijn_survey_1980}. The implementation of his Automath system is first motivated by describing how automated proof checking would benefit working mathematicians.
% it would increase confidence in the results; it would help in understanding proofs by requiring explicit assumptions; and it would allow for proofs to be processed like any other object in computer memory---one could, for example, find out whether a certain axiom or rule was used in a proof.

Under the heading \enquote{use of typing for reasoning}, de Bruijn gives some intuition for what he calls the \enquote{idea of \emph{propositions as types}} as applied to implication: given propositions \(p\) and \(q\), a proof of the implication \(p \rightarrow q\) can be interpreted as a procedure by which a proof of \(q\) is generated for any \(p\) passed to it. Later sections of the report discuss the potential benefits of introducing a proposition sort (\enquote{expression of degree 1}), and the beginnings of what would nowadays be called tactics. Both ideas have been realised in Coq, for example.\footnote{\url{http://coq.inria.fr}}

\begin{table}[h]
    \centering
    \begin{tabular}{l l}
        \toprule
        \textit{Logic} & \textit{Types} \\
        \midrule
	implication & \(\lambda\)-abstraction \\
	modus ponens & function application \\
	falsity & functions into the empty type \\
	conjunction & pairs \\
	disjunction & sum types \\
        \bottomrule
    \end{tabular}
    \caption{The Curry-Howard correspondence according to Howard \parencite{howard_formulae-as-types_1980}.}
    \label{tab:howard}
\end{table}


\section{More powerful type systems emerge}

\subsection{Martin-Löf's intuitionistic type theory}

A set of notes from a lecture series given by Per Martin-Löf in 1980 is the canonical reference\footnote{%
Martin-Löf published several papers on type theory and logic, as did de Bruijn, Coquand, Huet and most other authors whose publications this literature survey discusses. It is clear from the amount of cross-referencing and citing in those papers that the leading researchers at the time were often in close contact, criticising each other's work, finding mistakes or opening up new perspectives---\emph{science} was being done.

This makes untangling the development of the subject \emph{a posteriori} difficult. Martin-Löf developed his type theory over the course of multiple publications spanning decades. \enquote{The} calculus of constructions does not exists; instead, it is a term used by Coquand and Huet to refer to a family of type theories and conceptual programming languages described in at least four papers.

In this survey, I usually cite and describe the latest of such a series of publications on a particular topic by a particular author or set of authors.}
%
for \emph{intuitionistic type theory} (often referred to as \enquote{Martin-Löf type theory}), on which most of the following work discussed in this survey is based \autocite{sambin_intuitionistic_1984}.

Before going into type theory proper, Martin-Löf carefully distinguishes \emph{propositions} from \emph{judgements}: propositions are generated from logical operations and held to be true; holding a proposition to be true is to make a judgement. So when \(A\) is a proposition, then \enquote{\(A\) is true} is a judgement. The premisses and conclusion of a logical inference are judgements. As an example, the rule

\begin{prooftree}
\AxiomC{\(A\)}
\UnaryInfC{\(A \vee B\)}
\end{prooftree}

implicitly assumes that \(A\) and \(B\) are propositions, and states that we can infer that \(A \vee B\) holds if \(A\) is true. A formal rule would have to make this explicit:

\begin{prooftree}
\AxiomC{\(A\) prop}
\AxiomC{\(B\) prop}
\AxiomC{\(A\) true}
\TrinaryInfC{\(A \vee B\) true}
\end{prooftree}

\emph{Types} are then introduced using the following four judgements:

\begin{enumerate}
\item \(A\) is a set/type (in the notes this is written as \(A\) set). Sets/types are defined by describing how their canonical elements are formed and what equality of elements means in this set/type.
\item \(A\) and \(B\) are equal sets/types (\(A = B\)). \(A\) and \(B\) are equal if \(a : A\) entails \(a : B\) and \(a = b : A\) entails \(a = b : B\) and vice versa.
\item \(a\) is an element of the set/type \(A\) (written \(a \in A\) in the notes; we also use \(a : A\) here). This means that \(a\) is a \enquote{method (or program), which, when executed, yields a canonical element of \(A\) as a result.}
\item \(a\) and \(b\) are equal elements of the set/type \(A\) (written \(a = b \in A\) or \(a = b : A\)). For this to hold, \(a\) and \(b\) must yield equal canonical elements of \(A\) when executed.
\end{enumerate}

A discussion of \emph{equality} in the context of type theory is included. It differentiates between definitional (intensional) equality and propositional (extensional) equality. \emph{Definitional equality} (written \(a \equiv b\)) is the \enquote{equivalence relation generated by abbreviatory definitions, changes of bound variables and the principle of substituting equals for equals.} It is decidable on a meta-level, but \(a \equiv b\) is \emph{not} a proposition within the theory.

\emph{Propositional equality}, on the other hand, resides \emph{within} the theory. \(I(A, a, b)\) is a proposition asserting that \(a\) and \(b\) are equal elements of \(A\).

The remainder of the text contains a systematic account of the correspondence between ways of constructing types on the one hand and logical connectives and quantifiers on the other.\footnote{In the notes, set notation and nomenclature is used for types. I have tried to modernise the notation here for consistency.} Each type-forming symbol (see \cref{tab:martin-loef}) is defined by giving four rules:

\begin{description}
\item[Formation rules] state that this symbol forms a new type from existing types.
\item[Introduction rules] define the meaning of the symbol by giving its \emph{canonical elements}.
\item[Elimination rules] say how functions can be defined on this symbol's types.
\item[Equality rules] \enquote{relate the introduction and elimination rules by showing how a function defined by means of the elimination rule operates on the canonical elements of the [type] which are generated by the introduction rules}.
\end{description}

Considering these from the perspective of logic, formation rules are used to form propositions, introduction and elimination rules allow for Gentzen-style natural deduction, and equality rules correspond to reduction rules---in the parlance of \(\lambda\)-calculus, they would be called \(\beta\)- and \(\eta\)-reduction.

\begin{table}[h]
    \centering
    \begin{tabular}{l l l l}
        \toprule
        \textit{Logic} & & \textit{Types} & \\
        \midrule
	universal quantif'n & \((\forall x \in A)\;B(x)\)
		& generalised product & \((\Pi x : A)\;B(x)\) \\
	implication & \(A \rightarrow B\)
		& ---with \(B\) indep't of \(x\) & \((\Pi x : A)\;B\) \\
	existential quantif'n & \((\exists x \in A)\;B(x)\)
		& disjoint union & \((\Sigma x : A)\;B(x)\) \\
	conjunction & \(A \wedge B\)
		& ---with \(B\) indep't of \(x\) & \((\Sigma x : A)\;B\) \\
	disjunction & \(A \vee B\)
		& sum & \(A + B\) \\
	identity & \(A = B\)
		& equality & \(I(A, a, b)\) \\
	
        \bottomrule
    \end{tabular}
    \caption{Symbols defined in Martin-Löf's intuitionistic type theory \parencite{sambin_intuitionistic_1984}.}
    \label{tab:martin-loef}
\end{table}

\subsection{Coquand and Huet's calculus of constructions}

Coquand and Huet's \emph{calculus of constructions} blends together the theoretical work by Martin-Löf and Girard with the more concrete proposals of de Bruijn \autocite{coquand_calculus_1988}. They first describe the metatheory of the calculus and then provide suggestions concerning how such a calculus could be implemented in a user-friendly way.

In the calculus of constructions, \(*\) represents the universe of all types, that is, the type of all types \emph{and} the type of all propositions. \(*\) is not of type \(*\) to avoid Girard's paradox.\footnote{A judgement like \(* : *\) in a type theory makes it possible to prove any proposition, or equivalently, implies that every type (even \(\bot\)) is inhabited. This finding is usually referred to as \emph{Girard's paradox}, and the publication that is usually cited in this context is his doctoral thesis, \textcite{girard_interpretation_1972} (written in French). Thierry Coquand wrote a very readable introduction to the issue for the Stanford Encyclopedia of Philosophy, relating it, amongst others, to Russel's paradox \autocite{coquand_type_2014}. Coq and Agda avoid Girard's paradox via an infinite hierarchy of types---an approach that closely resembles Grothendieck's solution to similar issues in set theory by introducing what are now called \emph{Grothendieck universes} \autocite{artin_orie_1972}.} This differs from the Automath languages, where propositions are of type \texttt{prop}, types are of type \texttt{type} and both \texttt{type} and \texttt{prop} have type \(\tau\).

\emph{Contexts} (Martin-Löf's \enquote{hypothetical judgements}) are defined as \enquote{[t]erms formed solely of products over \(*\) [\dots] They are the types of logical propositions and proposition schemas}. The paper refers to other terms as \emph{objects}.

The core calculus, which resembles Martin-Löf's and also draws inspiration from Girard, is then extended by conversion rules of the form \(\Gamma \vdash M \cong N\) meaning that the terms \(M\) and \(N\) denote the same object (where \(\Gamma\) is a context). It formalises \(\beta\)-conversion at the type level.

An interpretation (or semantics) of terms is given: contexts map to products, for example, and variables to projections picking out the appropriate type from the context. It is then shown that the calculus of constructions is \emph{consistent} in the sense that \enquote{there exists a proposition which is not inhabited.} This non-inhabited proposition or empty type is usually referred to as \(\bot\) (\enquote{bottom}). What this theorem means is that there is no term in the calculus of constructions of type \(\bot\), or alternatively, \emph{false} cannot be derived.

The section concerned with making the calculus of constructions user-friendly makes three concrete proposals: allowing the local definitions (constants) via a \texttt{let} construct; making type arguments which can be inferred from other arguments implicit; and providing a way to extend the syntax of the language. As an example for the last suggestion, a composition operator is introduced as a syntax extension.

\section{Uniform formalisations of the Curry-Howard correspondence}
lala

\section{Recent developments}
lala

\section{Conclusions}
lala

\printbibliography

\end{document}
